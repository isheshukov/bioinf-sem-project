% Options for packages loaded elsewhere
\PassOptionsToPackage{unicode}{hyperref}
\PassOptionsToPackage{hyphens}{url}
%
\documentclass[
  russian,
  ignorenonframetext,
]{beamer}
\usepackage{pgfpages}
\setbeamertemplate{caption}[numbered]
\setbeamertemplate{caption label separator}{: }
\setbeamercolor{caption name}{fg=normal text.fg}
\beamertemplatenavigationsymbolsempty
% Prevent slide breaks in the middle of a paragraph
\widowpenalties 1 10000
\raggedbottom
\setbeamertemplate{part page}{
  \centering
  \begin{beamercolorbox}[sep=16pt,center]{part title}
    \usebeamerfont{part title}\insertpart\par
  \end{beamercolorbox}
}
\setbeamertemplate{section page}{
  \centering
  \begin{beamercolorbox}[sep=12pt,center]{part title}
    \usebeamerfont{section title}\insertsection\par
  \end{beamercolorbox}
}
\setbeamertemplate{subsection page}{
  \centering
  \begin{beamercolorbox}[sep=8pt,center]{part title}
    \usebeamerfont{subsection title}\insertsubsection\par
  \end{beamercolorbox}
}
\AtBeginPart{
  \frame{\partpage}
}
\AtBeginSection{
  \ifbibliography
  \else
    \frame{\sectionpage}
  \fi
}
\AtBeginSubsection{
  \frame{\subsectionpage}
}
\usepackage{lmodern}
\usepackage{amssymb,amsmath}
\usepackage{ifxetex,ifluatex}
\ifnum 0\ifxetex 1\fi\ifluatex 1\fi=0 % if pdftex
  \usepackage[T1]{fontenc}
  \usepackage[utf8]{inputenc}
  \usepackage{textcomp} % provide euro and other symbols
\else % if luatex or xetex
  \usepackage{unicode-math}
  \defaultfontfeatures{Scale=MatchLowercase}
  \defaultfontfeatures[\rmfamily]{Ligatures=TeX,Scale=1}
\fi
\usetheme[subsectionpage=simple,numbering=fraction,progressbar=frametitle,block=fill]{metropolis}
\usefonttheme{professionalfonts}
% Use upquote if available, for straight quotes in verbatim environments
\IfFileExists{upquote.sty}{\usepackage{upquote}}{}
\IfFileExists{microtype.sty}{% use microtype if available
  \usepackage[]{microtype}
  \UseMicrotypeSet[protrusion]{basicmath} % disable protrusion for tt fonts
}{}
\makeatletter
\@ifundefined{KOMAClassName}{% if non-KOMA class
  \IfFileExists{parskip.sty}{%
    \usepackage{parskip}
  }{% else
    \setlength{\parindent}{0pt}
    \setlength{\parskip}{6pt plus 2pt minus 1pt}}
}{% if KOMA class
  \KOMAoptions{parskip=half}}
\makeatother
\usepackage{xcolor}
\IfFileExists{xurl.sty}{\usepackage{xurl}}{} % add URL line breaks if available
\IfFileExists{bookmark.sty}{\usepackage{bookmark}}{\usepackage{hyperref}}
\hypersetup{
  pdftitle={Вывод демографических историй при помощи байесовской оптимизации},
  pdfauthor={Илья Шешуков},
  hidelinks,
  pdfcreator={LaTeX via pandoc}}
\urlstyle{same} % disable monospaced font for URLs
\newif\ifbibliography
\usepackage{longtable,booktabs}
\usepackage{caption}
% Make caption package work with longtable
\makeatletter
\def\fnum@table{\tablename~\thetable}
\makeatother
\usepackage{graphicx,grffile}
\makeatletter
\def\maxwidth{\ifdim\Gin@nat@width>\linewidth\linewidth\else\Gin@nat@width\fi}
\def\maxheight{\ifdim\Gin@nat@height>\textheight\textheight\else\Gin@nat@height\fi}
\makeatother
% Scale images if necessary, so that they will not overflow the page
% margins by default, and it is still possible to overwrite the defaults
% using explicit options in \includegraphics[width, height, ...]{}
\setkeys{Gin}{width=\maxwidth,height=\maxheight,keepaspectratio}
% Set default figure placement to htbp
\makeatletter
\def\fps@figure{htbp}
\makeatother
\usepackage[normalem]{ulem}
% Avoid problems with \sout in headers with hyperref
\pdfstringdefDisableCommands{\renewcommand{\sout}{}}
\setlength{\emergencystretch}{3em} % prevent overfull lines
\providecommand{\tightlist}{%
  \setlength{\itemsep}{0pt}\setlength{\parskip}{0pt}}
\setcounter{secnumdepth}{-\maxdimen} % remove section numbering
\usepackage{xcolor}
\usepackage{xspace}
\newcommand{\dadi}{∂a∂i\ }
\usepackage{unicode-math}
\usepackage{booktabs}
\ifxetex
  % Load polyglossia as late as possible: uses bidi with RTL langages (e.g. Hebrew, Arabic)
  \usepackage{polyglossia}
  \setmainlanguage[]{russian}
\else
  \usepackage[shorthands=off,main=russian]{babel}
\fi

\usepackage{animate}

\title{Вывод демографических историй при помощи байесовской оптимизации}
\author{Илья Шешуков}
\date{}

\begin{document}
\frame{\titlepage}

\hypertarget{ux432ux432ux435ux434ux435ux43dux438ux435}{%
\section{Введение}\label{ux432ux432ux435ux434ux435ux43dux438ux435}}

\begin{frame}{Демографическая модель популяции}
\protect\hypertarget{ux434ux435ux43cux43eux433ux440ux430ux444ux438ux447ux435ux441ux43aux430ux44f-ux43cux43eux434ux435ux43bux44c-ux43fux43eux43fux443ux43bux44fux446ux438ux438}{}

Имея геномы людей, хотим понять как изменялись их популяции. Как
менялась численность, когда популяции разделялись, как сильно они
мигрировали.

\begin{figure}
\centering
\includegraphics[width=\textwidth,height=0.4\textheight]{./pics/outofafrica.png}
\caption{Демографическая модель африканского происхождени человека}
\end{figure}

\end{frame}

\begin{frame}{Аллель-частотный спектр}
\protect\hypertarget{ux430ux43bux43bux435ux43bux44c-ux447ux430ux441ux442ux43eux442ux43dux44bux439-ux441ux43fux435ux43aux442ux440-1}{}

Аллель-частотный спектр это распределение частоты аллелей в данных
локусах в популяции или выборке.

\begin{figure}
\centering
\includegraphics[width=0.5\textwidth,height=\textheight]{./pics/sfs.png}
\caption{График АЧС}
\end{figure}

\end{frame}

\begin{frame}{Пример}
\protect\hypertarget{ux43fux440ux438ux43cux435ux440}{}

\begin{longtable}[]{@{}lllllllll@{}}
\toprule
& SNP 1 & SNP 2 & SNP 3 & SNP 4 & SNP 5 & SNP 6 & SNP 7 & SNP
8\tabularnewline
\midrule
\endhead
& 0 & 1 & 0 & 0 & 0 & 0 & 1 & 0\tabularnewline
& 1 & 0 & 1 & 0 & 0 & 0 & 1 & 0\tabularnewline
& 0 & 1 & 1 & 0 & 0 & 1 & 0 & 0\tabularnewline
& 0 & 0 & 0 & 0 & 1 & 0 & 1 & 1\tabularnewline
& 0 & 0 & 1 & 0 & 0 & 0 & 1 & 0\tabularnewline
& 0 & 0 & 0 & 1 & 0 & 1 & 1 & 0\tabularnewline
Сумма & 1 & 2 & 3 & 1 & 1 & 2 & 5 & 1\tabularnewline
\bottomrule
\end{longtable}

Спектр: \(\begin{pmatrix}4&2&1&0&1\end{pmatrix}\)

\end{frame}

\hypertarget{ux43aux430ux43a-ux44dux442ux43e-ux434ux435ux43bux430ux435ux442ux441ux44f-ux441ux435ux439ux447ux430ux441}{%
\section{Как это делается
сейчас}\label{ux43aux430ux43a-ux44dux442ux43e-ux434ux435ux43bux430ux435ux442ux441ux44f-ux441ux435ux439ux447ux430ux441}}

\begin{frame}{∂a∂i\ }
\protect\hypertarget{section}{}

\url{https://bitbucket.org/gutenkunstlab/dadi/}

\begin{itemize}
\tightlist
\item
  Плюсы

  \begin{itemize}
  \tightlist
  \item
    Она работает
  \item
    Ей пользуются реальные люди
  \end{itemize}
\item
  Минусы

  \begin{itemize}
  \tightlist
  \item
    Решает дифференциальное уравнение в частных производных, что долго
  \item
    Использует методы локальной оптимизации, что малоэффективно
  \item
    Для работы необходимо руками писать Питон
  \end{itemize}
\end{itemize}

\end{frame}

\begin{frame}{moments}
\protect\hypertarget{moments}{}

\url{https://bitbucket.org/simongravel/moments}

\begin{itemize}
\tightlist
\item
  Плюсы

  \begin{itemize}
  \tightlist
  \item
    Эффективнее, чем ∂a∂i\ , особенно на больших популяциях
  \end{itemize}
\end{itemize}

\end{frame}

\begin{frame}{GADMA}
\protect\hypertarget{gadma}{}

\url{https://github.com/ctlab/GADMA}

\begin{itemize}
\tightlist
\item
  Основана на ∂a∂i\ и moments
\item
  Использует генетический алгоритм для поиска значения параметров
  демографической модели
\item
  Не требует человеческого вмешательства
\end{itemize}

\end{frame}

\begin{frame}[standout]{Что можно сделать}
\protect\hypertarget{ux447ux442ux43e-ux43cux43eux436ux43dux43e-ux441ux434ux435ux43bux430ux442ux44c}{}

Заменим генетический алгоритм байесовской оптимизацей.

\end{frame}

\begin{frame}{Байесовская оптимизация}
\protect\hypertarget{ux431ux430ux439ux435ux441ux43eux432ux441ux43aux430ux44f-ux43eux43fux442ux438ux43cux438ux437ux430ux446ux438ux44f}{}

\begin{itemize}
\tightlist
\item
  Алгоритм глобальной оптимизации
\item
  Хорошо работает для сложновычислимых функций (например, если нужно
  решать уравнение в частных производных), т.е. хорошо подходит для
  задачи
\item
  Можно параллелить
\item
  Менее эвристична, чем генетический алгоритм
\end{itemize}

\end{frame}

\begin{frame}{Красивые графики}
\protect\hypertarget{ux43aux440ux430ux441ux438ux432ux44bux435-ux433ux440ux430ux444ux438ux43aux438}{}

\includegraphics[width=0.4\textwidth,height=\textheight]{./pics/bayes.png}
~\includegraphics[width=0.5\textwidth,height=\textheight]{./pics/conv.png}

\end{frame}

\hypertarget{ux440ux435ux437ux443ux43bux44cux442ux430ux442ux44b}{%
\section{Результаты}\label{ux440ux435ux437ux443ux43bux44cux442ux430ux442ux44b}}

\begin{frame}{Планы (промежуточная презентация)}
\protect\hypertarget{ux43fux43bux430ux43dux44b-ux43fux440ux43eux43cux435ux436ux443ux442ux43eux447ux43dux430ux44f-ux43fux440ux435ux437ux435ux43dux442ux430ux446ux438ux44f}{}

\begin{itemize}
\tightlist
\item
  Заменить в ∂a∂i\ алгоритм градиентного спуска нa байесовскую
  оптимизацию.
\item
  Посмотреть станет ли лучше
\item
  Интегрировать в GADMA
\end{itemize}

\end{frame}

\begin{frame}{Реальность}
\protect\hypertarget{ux440ux435ux430ux43bux44cux43dux43eux441ux442ux44c}{}

\begin{itemize}
\tightlist
\item[$\boxtimes$]
  Заменить в \sout{∂a∂i\ } moments алгоритм градиентного спуска нa
  байесовскую оптимизацию.
\item[$\boxtimes$]
  Посмотреть станет ли лучше
\item[$\square$]
  Интегрировать в GADMA
\end{itemize}

\end{frame}

\begin{frame}{Сравнительная таблица}
\protect\hypertarget{ux441ux440ux430ux432ux43dux438ux442ux435ux43bux44cux43dux430ux44f-ux442ux430ux431ux43bux438ux446ux430}{}

\begin{longtable}[]{@{}lllll@{}}
\toprule
Данные & Оптимум & dadi & moments & GPyOpt\tabularnewline
\midrule
\endhead
\vtop{\hbox{\strut \textbf{2 популяции}}\hbox{\strut 6 переменных}} &
1066.823 & - & - & \vtop{\hbox{\strut 56 часов}\hbox{\strut f(x) =
1066.954}}\tabularnewline
\vtop{\hbox{\strut \textbf{2 популяции}}\hbox{\strut 8 переменных}} &
1070.048 & - & - & \vtop{\hbox{\strut 24 часа}\hbox{\strut f(x) =
1160.432}}\tabularnewline
\vtop{\hbox{\strut \textbf{3 популяции}}\hbox{\strut 13 переменных}} &
6316.578 & - & - & \vtop{\hbox{\strut 73 часа}\hbox{\strut f(x) =
7377.065}}\tabularnewline
\bottomrule
\end{longtable}

\end{frame}

\begin{frame}{Torque Generating Mechanism}
  \animategraphics[autoplay,loop,width=\linewidth]{2}{anim/}{1}{17}
\end{frame}


\begin{frame}[standout]{Конец}
\protect\hypertarget{ux43aux43eux43dux435ux446}{}

Спасибо за внимание

TODO

\begin{itemize}
\item
  время -\textgreater{} итерации
\item
  анимированные графики в презентации
\item
  убрать дади, получить данные по моментс
\item
  графики сходимости по
\item
  добавить лирики (что происходилов работе !!!!)
\item
  сравнить на других данных
\item
  что бы ещё можно было сделать с проектом интегрироваться в гадму
\end{itemize}

\end{frame}

\end{document}
